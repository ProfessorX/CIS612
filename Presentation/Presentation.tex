
%%% Local Variables: 
%%% mode: latex
%%% TeX-master: t
%%% End: 
\documentclass[hyperref=true]{beamer}

% Package and Theme
% Some more packages will be used.
\usepackage{amsmath}
\usepackage{amssymb}
\usepackage{algorithmic}
\usetheme{AnnArbor}
\graphicspath{{Figure/}{figures/}{figure/}{pictures/}{picture/}{pic/}{pics/}}


% Make preparation. Make slides. And create wealth by yourself.
% Presentation. Homework. Software Engineering reading.

\begin{document}
\title[Change the World]{How to Change the World with Donald Knuth}
\author{Abraham Xiao}
\institute[Masdar Institute]{Masdar Institute of Science and
  Technology}
\date[CIS612 Presentation]{Information Security Project Presentation}



\begin{frame}
  \titlepage{}
\end{frame}

\begin{frame}
  \tableofcontents{}
\end{frame}

\section{Discrete Logarithm}
\label{sec:discrete-logarithm}



\begin{frame}
  \frametitle{Discrete Logarithm in a Nutshell}
The security of many cryptographic techniques depends on the
intractability of discrete logarithm problem.\\[4pt]A partial list of these
include:
\begin{itemize}
\item Diffie–Hellman key agreement and its derivatives.
\item ElGamal encryption.
\item ElGamal signature scheme and its variants.
\end{itemize}
General setting for algorithms in this section are:
\begin{itemize}
\item A (multiplicatively written) finite cyclic group $G$
\item $n$ is the order of group $G$
\item $\alpha$ is a generator of group $G$\footnote{For more math
    background, refer to~\cite{Rosen:2012}.}
\end{itemize}
\end{frame}

\begin{frame}
  \frametitle{Relevant Definitions}
Cyclic group and its generator.
  \begin{definition}
    A group is \emph{cyclic} if there is an element $\alpha\in G$ such
    that for each $b\in G$ there is an integer $i$ with
    $b=\alpha^{i}$. Such an element $\alpha$ is called a generator of $G$.
  \end{definition}
Discrete logarithm.
  \begin{definition}
    Let $G$ be a finite cyclic group of order $n$. Let $\alpha$ be a
    generator of $G$, and let $\beta\in G$. The \emph{discrete
      logarithm of $\beta$ to the base $\alpha$}, denoted
    $\log_{\alpha}\beta$, is the unique integer $x$, $0\leq x\leq
    n-1$, such that $\beta=\alpha^{x}$\cite{Menezes:1996:HAC:548089}.
  \end{definition}
\end{frame}


\begin{frame}
  \frametitle{A Discrete Logarithm Example}
  \begin{example}
    Let $p=97$. Then $\mathbb{Z}_{97}^{*}$ is a cyclic group of order
    $n=96$. A generator of $\mathbb{Z}_{97}^{*}$ is $\alpha=5$. Since
    $5^{32}\equiv 35\mod 97$, $\log_{5}35 =32$ in $\mathbb{Z}_{97}^{*}$.
  \end{example}
\end{frame}

% http://v.youku.com/v_show/id_XNjQ0MzYwMDMy.html 
% Add this link to a possible place
\begin{frame}
  \frametitle{The Diffie–Hellman Problem}
The Diffie–Hellman problem is closely related to the well-studied
discrete logarithm problem.
\begin{definition}
  The \emph{Diffie–Hellman problem} is the following: given a prime
  $p$, a generator $\alpha$ of $\mathbb{Z}_{p}^{*}$, and elements
  $\alpha^{a}\mod p$ and $\alpha^{b}\mod p$, find $\alpha^{ab}\mod p$.
\end{definition}
% \\[4pt]
Wait! Could we just possibly do
\begin{equation}
  \label{eq:diffie–hellman-1}
  \alpha^{a}\times\alpha^{b}\rightarrow\alpha^{ab}
\end{equation}
Well, life is not as easy as it looks like\ldots
\begin{equation}
  \label{eq:diffie–hellman-2}
  \alpha^{a}\times\alpha^{b}=\alpha^{a+b}
\end{equation}

\end{frame}

\begin{frame}
  \frametitle{Links between Discrete Logarithm and Diffie–Hellman
    Problem}
\textbf{Suppose} that the discrete logarithm problem in
$\mathbb{Z}_{p}^{*}$ could be efficiently solved\footnote{In math,
  the assumption is as important as,\\ if not more important than
  induction in many situations.}. Then given $\alpha$, $p$,
$\alpha^{a}\mod p$ and $\alpha^{b}\mod p$, one could first find $a$
from $\alpha$, $p$ and $\alpha^{a}\mod p$ by
what?!\\[8pt]\textbf{Solving a discrete logarithm problem}, and  then
compute ${(\alpha^{b})}^{a}=\alpha^{ab}\mod p$.

\end{frame}

\section{ElGamal Cryptosystem}
\label{sec:elgamal-cryptosystem}



\begin{frame}
  \frametitle{ElGamal public-key encryption}
The ElGamal public-key encryption scheme can be viewed as
Diffie–Hellman key agreement\footnote{Yet another fancy nickname for
  key exchange} in key transfer mode.\\
Its security is based on the intractability of the discrete logarithm
problem (Section~\ref{sec:discrete-logarithm}) and the Diffie–Hellman 
problem (Section~\ref{sec:elgamal-cryptosystem}).
\end{frame}


\begin{frame}
  \begin{figure}
    \centering
  \begin{algorithmic}
    \ENSURE~A public key and its corresponding private key is created
    for every entity.
    \STATE~Steps to generate key pairs are described as follows:
    \begin{enumerate}
    \item Generate a prime $p$ that is large enough and cannot be
      predicted, i.e.\ it should be generated randomly. Find a
      generator $\alpha$ of the multiplicative group $\mathbb{Z}_{p}^{*}$ of
      integers modulo $p$.
    \item Randomly select an integer $a$ satisfying $1\leq
      a\leq p-2$. Then calculate $\alpha^{a}\mod p$.
    \item The public key is returned as $(p,\alpha,\alpha^{a})$; The
      private key is returned as $a$.
    \end{enumerate}
  \end{algorithmic}
  \caption{\textbf{Algorithm} Key generation for ElGamal public-key encryption}
  \label{fig:basic-elgamal-encryption}
  \end{figure}
\end{frame}





\begin{frame}[allowframebreaks]
  \frametitle{References}
\bibliographystyle{amsalpha}
\bibliography{Ref}

\end{frame}


\end{document}
