
%%% Local Variables: 
%%% mode: latex
%%% TeX-master: t
%%% End: 
\documentclass[hyperref=true]{beamer}

% Package and Theme
% Some more packages will be used.
\usepackage{amsmath}
\usepackage{amssymb}
\usepackage{algorithmic}
\usetheme{AnnArbor}
\graphicspath{{Figure/}{figures/}{figure/}{pictures/}{picture/}{pic/}{pics/}}


% Make preparation. Make slides. And create wealth by yourself.
% Presentation. Homework. Software Engineering reading.

\begin{document}
\title[Change the World]{How to Change the World with Donald Knuth}
\author{Abraham Xiao}
\institute[Masdar Institute]{Masdar Institute of Science and
  Technology}
\date[CIS612 Presentation]{Information Security Project Presentation}



\begin{frame}
  \titlepage{}
\end{frame}

\begin{frame}
  \tableofcontents{}
\end{frame}

\section{Discrete Logarithm}




\begin{frame}
  \frametitle{Discrete Logarithm in a Nutshell}
The security of many cryptographic techniques depends on the
intractability of discrete logarithm problem.\\[4pt]A partial list of these
include:
\begin{itemize}
\item Diffie–Hellman key agreement and its derivatives.
\item ElGamal encryption.
\item ElGamal signature scheme and its variants.
\end{itemize}
General setting for algorithms in this section are:
\begin{itemize}
\item A (multiplicatively written) finite cyclic group $G$
\item $n$ is the order of group $G$
\item $\alpha$ is a generator of group $G$\footnote{For more math
    background, refer to~\cite{Rosen:2012}.}
\end{itemize}
\end{frame}

\begin{frame}
  \frametitle{Relevant Definitions}
Cyclic group and its generator.
  \begin{definition}
    A group is \emph{cyclic} if there is an element $\alpha\in G$ such
    that for each $b\in G$ there is an integer $i$ with
    $b=\alpha^{i}$. Such an element $\alpha$ is called a generator of $G$.
  \end{definition}
Discrete logarithm.
  \begin{definition}
    Let $G$ be a finite cyclic group of order $n$. Let $\alpha$ be a
    generator of $G$, and let $\beta\in G$. The \emph{discrete
      logarithm of $\beta$ to the base $\alpha$}, denoted
    $\log_{\alpha}\beta$, is the unique integer $x$, $0\leq x\leq
    n-1$, such that $\beta=\alpha^{x}$\cite{Menezes:1996:HAC:548089}.
  \end{definition}
\end{frame}






\begin{frame}[allowframebreaks]
  \frametitle{References}
\bibliographystyle{amsalpha}
\bibliography{Ref}

\end{frame}


\end{document}
