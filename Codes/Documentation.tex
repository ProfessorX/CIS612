
%%% Local Variables: 
%%% mode: latex
%%% TeX-master: t
%%% End: 
\documentclass[12pt]{article}


% Packages
\usepackage{amsmath}
\usepackage{amssymb}
\usepackage{graphicx}
\usepackage{hyperref}


% Some house-keeping
\newtheorem{Definition}{Definition}

\begin{document}

\title{Simple Documentation for RSA Implementation}
\author{Abraham Xiao}
\maketitle

\section{Introduction}
\label{sec:introduction}

For convenience, we cite some facts and description
from~\cite{Menezes:1996:HAC:548089} without much more mentioning. We
hope this will not intrigue intelligence property issues. 
\par
\begin{Definition}
  The \emph{RSA problem} is the following: given a positive integer
  $n$ that is a product of two distinct odd primes $p$ and $q$, a
  positive integer $e$ such that $gcd(e,(p-1)(q-1))=1$, and an integer
  $c$, find an integer $m$ such that $m^{e}\equiv c\mod n$.
\end{Definition}
In other words, the RSA problem is that of finding $e^{th}$ roots
modulo a composite integer $n$. The condition imposed on the problem
parameters $n$ and $e$ ensure that for each integer
$c\in{0,1,\ldots,n-1}$ there is exactly one $m\in{0,1,\ldots,n-1}$
such that $m^{e}\equiv c\mod n$. Equivalently, the function
$f:\mathbb{Z}_{n}\longrightarrow\mathbb{Z}_{n}$ defined as
$f(m)=m^{e}\mod n$ is a \emph{permutation}.

\section{Implementation}
\label{sec:implementation}

\subsection{Data Structure}
\label{sec:data-structure}

A special data structure containing two primes $p$ and $q$, the
multiplication of $(p-1)(q-1)$ as well as public, private key pairs is
defined as follows:
\begin{verbatim}
typedef struct RSA_PARAM_Tag
{
	unsigned __int64 p, q;  // p and q are two primes
	unsigned __int64 f;	// f=(p-1)*(q-1)
	unsigned __int64 n, e;	// n=pq; gcd(e,f)=1 public keys
	unsigned __int64 d;	// private key, ed=1(mod f), gcd(n,d)=1
}RSA_PARAM;
\end{verbatim}
A class containing a private data as well as public data, method is
defined as follows:
\begin{verbatim}
class RandNumber
{
private:
	unsigned __int64 randSeed;
public:
	RandNumber(unsigned __int64 s = 0);
	unsigned __int64 Random(unsigned __int64 n);
};
\end{verbatim}

For the rest part we itemize features in our implementation.
\begin{itemize}
\item An array of small prime table is created to speed-up the process
  of identifying if a large number is a prime or composite.
\item The seed used to generate large random number is taken from
  current calendar time to ensure enough randomness. 
\item A random number is generated in a way of multiplying a large
  enough number and then add another one.
\item Rabin-Miller primality test is implemented. And the testing loop
  is adjustable.
\item Both the Euclidean algorithm and binary algorithm for
  calculating \emph{greatest common divisor} are implemented.
\item The whole RSA algorithm is implemented neatly.
\end{itemize}


\section{Samples}
\label{sec:samples}

We use a toy sample to conclude this simple documentation. Up to now,
the string with spaces is not supported. We are sorry for that,
indeed.
\begin{verbatim}
abrahamx91@debian:~/Professional/Git/CIS612-Composition/Codes$
 ./a.out 
p=47911
q=38839
f=(p-1)*(q-1)=1860728580
n=p*q=1860815329
e=46387
d=1574922403

 Please enter your plaintext: Abraham-Xiao-Keep-Moving!

 Ciphertext is: b58c31a 6d4c7761 15dafa09 17a7e101 2c02bb80 
 17a7e101 650e1f0c 64dc1f07 2c3b1738 1189bc8c 17a7e101 19873f79 
 64dc1f07 5596ced9 38a8ee68 38a8ee68 9bb7fbf 64dc1f07 49bec0cc 
 19873f79 52d47daf 1189bc8c 2dd5496b 13442502 2bec903d 0 

Decipher:  You plaintext should be: Abraham-Xiao-Keep-Moving!

abrahamx91@debian:~/Professional/Git/CIS612-Composition/Codes$ 

\end{verbatim}
Some parts are manually modifies due to page space issues.

\section*{Acknowledgment}
\label{sec:acknowledgment}
I would like to thank Dr.\ Zeyar for preparing high quality lectures
throughout the whole Fall 2013 semester. In addition, I am extremely
grateful for being able to carry out \emph{care-free} research at
Masdar Institute of Science and Technology, especially as a late
applicant last year\footnote{I submitted my full application just 2 weeks
before the deadline. But I got the offer pretty fast.}.



\bibliographystyle{acm}
\bibliography{Ref}


\end{document}
