
%%% Local Variables: 
%%% mode: latex
%%% TeX-master: t
%%% End: 
\documentclass[twocolumn]{article}



% My mom is always telling me that, a man has to learn to do some
% basic house keeping. Or he shall never turn a good man.
\usepackage{amsmath}
\usepackage{amssymb}
\usepackage[margin=10mm]{geometry}
\usepackage{graphicx}
\usepackage{algorithm}
\usepackage{algorithmic}


\begin{document}
\begin{itemize}
\item \textbf{CIA}, a modern definition. Confidentiality: prevent
  unauthorized reading of information. Integrity: detect unauthorized
  writing of information. Availability: data is available in a timely
  manner when needed. 
\item \textbf{Network Security}. Various protocols play a critical
  role, and cryptography matters a lot in protocol (especially network
  protocols) design and analysis.
\item \textbf{Kerckhoof's Principle}. The system is completely known
  to the attacker; only the key is secret; the crypto algorithms are
  not secret. 
\item \textbf{Confusion and Diffusion}. Confusion: obscuring the
  relationship between plaintext and ciphertext. Diffusion: spreading
  the plaintext statistics through the ciphertext. A little note: hash
  function can be viewed as \emph{one way cryptography}.
\item \textbf{Block Cipher}. It's really just an ``electronic''
  version of a codebook, and employs both confusion and diffusion. 
  % \begin{algorithm}
  %   \begin{algorithmic}
  %     \STATE~Life is a bitch.
  %     \IF{You are a bitch}
  %     \STATE~Do something never that silly.
  %     \ENDIF
  %   \end{algorithmic}
  % \end{algorithm}
  \begin{algorithm}
    \caption{RC4 Keystream Byte}
    \label{algo:rc4-keystream-byte}
    \begin{algorithmic}
      \STATE~$i=(i+1)\mod 256$
      \STATE~$j=(j+S[i]\mod 256)$
      \STATE~swap $(S[i], S[j])$
      \STATE~$t=(S[i]+S[j]\mod 256)$
      \STATE~$Keystream~ byte=S[t]$
    \end{algorithmic}
  \end{algorithm}
\item \textbf{Feistel Cipher}. It's a general cipher design
  principle. $L_{i}=R_{i-1}$ and $R_{i}=L_{i-1}\oplus
  F(R_{i-1},K_{i})$.
\item \textbf{DES}. The security of this cryptosystem has much to do
  with \emph{S-box}. Steps: an initial permutation before round 1;
  halves are swapped after last round; a final permutation applied to
  $R_{16},L_{16}$. 
\item 
\end{itemize}
\end{document}
