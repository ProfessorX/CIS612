
%%% Local Variables: 
%%% mode: latex
%%% TeX-master: t
%%% End: 
\documentclass[twocolumn]{article}



% My mom is always telling me that, a man has to learn to do some
% basic house keeping. Or he shall never turn a good man.
\usepackage{amsmath}
\usepackage{amssymb}
\usepackage[margin=10mm]{geometry}
\usepackage{graphicx}
\usepackage{algorithm}
\usepackage{algorithmic}


\begin{document}
\begin{itemize}
\item \textbf{CIA}, a modern definition. Confidentiality: prevent
  unauthorized reading of information. Integrity: detect unauthorized
  writing of information. Availability: data is available in a timely
  manner when needed. 
\item \textbf{Network Security}. Various protocols play a critical
  role, and cryptography matters a lot in protocol (especially network
  protocols) design and analysis.
\item \textbf{Kerckhoof's Principle}. The system is completely known
  to the attacker; only the key is secret; the crypto algorithms are
  not secret. 
\item \textbf{Confusion and Diffusion}. Confusion: obscuring the
  relationship between plaintext and ciphertext. Diffusion: spreading
  the plaintext statistics through the ciphertext. A little note: hash
  function can be viewed as \emph{one way cryptography}.
\item \textbf{Stream Cipher}. Both A5/1 and RC4 are examples of this
  symmetric cryptosystem. It generalized the idea of a one-time pad,
  except that we trade provably security with a relatively small (and
  manageable) key. The key is stretched into a long stream of bits,
  which is then used just like a one-time pad. 
\item \textbf{Block Cipher}. It's really just an ``electronic''
  version of a codebook, and employs both confusion and diffusion. 
  % \begin{algorithm}
  %   \begin{algorithmic}
  %     \STATE~Life is a bitch.
  %     \IF{You are a bitch}
  %     \STATE~Do something never that silly.
  %     \ENDIF
  %   \end{algorithmic}
  % \end{algorithm}
  \begin{algorithm}
    \caption{RC4 Keystream Byte}
    \label{algo:rc4-keystream-byte}
    \begin{algorithmic}
      \STATE~$i=(i+1)\mod 256$
      \STATE~$j=(j+S[i]\mod 256)$
      \STATE~swap $(S[i], S[j])$
      \STATE~$t=(S[i]+S[j]\mod 256)$
      \STATE~$Keystream~ byte=S[t]$
    \end{algorithmic}
  \end{algorithm}
\item \textbf{Feistel Cipher}. It's a general cipher design
  principle. $L_{i}=R_{i-1}$ and $R_{i}=L_{i-1}\oplus
  F(R_{i-1},K_{i})$.
\item \textbf{DES}. The security of this cryptosystem has much to do
  with \emph{S-box}. Steps: an initial permutation before round 1;
  halves are swapped after last round; a final permutation applied to
  $R_{16},L_{16}$. 
  \begin{algorithm}
    \caption{TEA Encryption}
    \label{algo:tea-encryption}
    \begin{algorithmic}
      \STATE~$(K[0],K[1],K[2],K[3])=128~bit~key$
      \STATE~$(L,R)=plaintext$ ($64-bit$ block)
      \STATE~$delta=0$x$9e3779b9$
      \STATE~$sum=0$
      \FOR{$i=1$ to $32$}
      \STATE~$sum=sum+delta$
      \STATE~$L=L+(((R\ll 4)\oplus K[0])\oplus (R+sum)\oplus ((R\gg
      5)\oplus K[1]))$
      \STATE~$R=L+(((L\ll 4)\oplus K[2])\oplus (L+sum)\oplus ((L\gg
      5)\oplus K[3]))$
      \STATE~next $i$
      \ENDFOR
      \STATE~$ciphertext=(L,R)$
    \end{algorithmic}
  \end{algorithm}
\item \textbf{Block Cipher Modes}. ECB:\@ encrypt each block
  independently. CBC:\@ chain the blocks together. For this mode, a
  random initialization vector is required. CTR:\@ block cipher acts
  like stream one.  
\item \textbf{Data Integrity}. The encryption process does provide
  confidentiality, but no guarantee of integrity. 
  \begin{algorithm}
    \begin{algorithmic}
      \caption{Key generation for RSA public key encryption}
      \label{algo:keygen-for-rsa-encrypt}
      \ENSURE~Each entity creates an RSA public key and a
      corresponding private key. Each entity A should do the
      following:
      \begin{enumerate}
      \item Generate two large random and distinct primes $p$ and $q$,
        each roughly the same size.
      \item Compute $n=pq$ and $\phi=(p-1)(q-1)$.
      \item Select a random integer $e$, $1\leq e\leq\phi$, such that
        $\gcd(e,\phi)=1$. 
      \item Use the extended Euclidean algorithm to compute the unique
        integer $d$, such that $ed\equiv 1\mod\phi$.
      \item A's public key is $(n,e)$, private key is $d$.
      \end{enumerate}
    \end{algorithmic}
  \end{algorithm}
\item \textbf{RSA Validity Proof}.
  \begin{itemize}
  \item Since $ed\equiv 1\mod\phi$, there exists an integer $k$ such
    that $ed=1+k\phi$. 
  \item Now if $\gcd(m,p)=1$, then by Fermat's theorem, $m^{p-1}\equiv
    1\mod p$. 
  \item Raising both sides of this congruence to the power $k(q-1)$
    and then multiplying both sides by $m$ yields
    $m^{1+k(p-1)(q-1)}\equiv m\mod p$.
  \item On the other hand if $\gcd(m,p)=p$, then this last congruence
    is valid since each side is congruence to $0\mod p$. 
  \item Hence, in all cases, $m^{ed}\equiv m\mod p$. By the same
    argument, $m^{ed}\equiv m\mod q$. 
  \item Finally, since $p$ and $q$ are distinct primes, it follows
    that $m^{ed}\equiv m\mod n$. And hence, $c^{d}\equiv
    {(m^{e})}^{d}\equiv m\mod n$.
  \end{itemize}
\item \textbf{Cube Root attack} on RSA.\@ A simple but practical way to
  prevent is to pad message with random bits.
\item \textbf{Cryptographic Hash Function}. This function must provide
  the following:
  \begin{itemize}
  \item Compression. For any size input $x$, the output length,
    i.e. $h(x)$ is small. Usually a fixed length is pre-defined.
  \item Efficiency. It must be easy to compute $h(x)$ for any input
    $x$. 
  \item One way. Given any value $y$, it's computationally infeasible
    to find a value $x$ such that $h(y)=x$.
  \item Weak Collision Resistance. Given $x$ and $h(x)$, it's
    infeasible to find any $y$, with $y\neq x$, such that
    $h(y)=h(x)$.
  \item Strong Collision resistance. It's (and should be so)
    infeasible to find any $x\neq y$ such that $h(x)=h(y)$.
  \end{itemize}
\item \textbf{Birthday Problem}. Strong one. How large much the $N$ be
  before the probability that someone shares the same birthday with
  me? Weak one. How many people must be in a room before the
  probability of at least two share the same birthday is larger than
  $0.5$?  
\item \textbf{Access Control}. Two easy-to-understand
  comparisons. Authentication: are you who you say you are?
  Authorization: are you allowed to do that fucking (forgive my
  rudeness; I am tired.) stuff?
\item \textbf{Common Attacks on Passwords}. Usually, this applies to
  many other similar stuff in security as well. Outsider, and then you
  ``act as if'' you are a normal user. Some time when ``the day''
  comes, you may have the privilege to ``self-promoting'' to (one of)
  the administrators.
\item \textbf{Iris Scan Attacks}. One thing pointed out in the slides,
  ``scanners could use light to'' make sure that it's scanning a live
  eye. 
\item \textbf{Web Cookies}. According to our official textbook, web
  cookies have ``some interesting security implications''. Web cookie
  is simply a numerical value that is stored and managed by one's
  browser. The website that one is visiting also stores the cookie,
  which is used to index a database that retains information about
  Alice (in textbook flavor). In a slightly stronger ``expression'', a
  password is used to initially authenticate Alice, after which the
  cookie is considered sufficient.
\item \textbf{Evaluation Assurance Level}. Note that a product with a
  higher EAL does not necessarily mean it \emph{does} possess a higher
  security power (forgive me for the lack of words). For example,
  suppose that product A is certificated EAL4, while product B carries
  EAL5 rating. All it means is that product A was evaluated for EAL4
  (and passed), while product B was actually evaluated for EAL5 (and,
  at the same time, passed). It is possible that product A could have
  achieved EAL5 or higher, but the developers simply felt it was not
  worth the cost and effort to try a higher EAL.\@
\item \textbf{Classification and Clearances}. Classification applies
  to \emph{objects}. Clearance applies to \emph{subjects}. 
\item \textbf{Compartments}. They serve to enforce the \emph{need to
    know} principle, that is, subjects are only allowed to know the
  information that they \emph{must} know for their work.
\item \textbf{Covert Channel}. Three things are required for a covert
  channel to exist. First, the sender and receiver must have access to
  a shared resource. Second, the sender must be able to vary some
  property of the shared resource that the receiver can
  observe. Finally, the sender and receiver must be able to
  synchronize their communication. 
\end{itemize}
\end{document}
     


